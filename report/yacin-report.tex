\documentclass{article}
\usepackage[pdftex]{graphicx}
% \usepackage{xcolor}
% \usepackage{latexsym}
% \usepackage{amssymb}
% \usepackage{fullpage}
% \usepackage{enumerate}
% \usepackage{setspace}
% \doublespacing

%% CS 536 Stuff
% \newcommand{\liff}{\leftrightarrow}
% \newcommand{\holds}{\vDash}
% \newcommand{\nholds}{\nvDash}

%% No paragraph indent
% \setlength{\parindent}{0pt} 

%% Paragraph ``skip'' settings
% \setlength{\parskip}{1ex plus 0.5ex minus 0.2ex}

% Title
\title{CS 541 - Scheme Compiler Report}
\author{Yacin Nadji}
\date{\today}

\begin{document}
\maketitle

I implemented through part of closures alongside Jordan, Tristan and myself. We began writing our own compilers in the beginning, but once we started reaching the more complicated portions we began to work together more closely. We would pick and choose from implementations between the three of us. We got most of the way through closures and tail recursion, but the busy semester closing halted progress preventing completion. I think small groups are the best way to work on this particular assignment, shown by the initial speedy progress we had and the amount accomplished by Alana/Johnny/et. al. when they got together to work on implementing functions.

Overall, I think it's a reasonable way to teach compilers, although I don't have a traditional compilers course to compare it to. I feel that Scheme was probably the best choice, so students don't have to worry about writing a tokenizer/parser before getting to the nitty gritty of writing a compiler. I think having a fully written compiler at each step would be useful for a class situation, in the event students fall behind and want to continue with penalty. Also, more ``correct'' instructions would be helpful; on more than one occasion I had to go against the directions to get a working compiler. I think supplementary reading and theoretical lectures should accompany the semester-long lab to further solidify the concepts. I definitely enjoyed the course, and I feel as though I learned a great deal.

\end{document}